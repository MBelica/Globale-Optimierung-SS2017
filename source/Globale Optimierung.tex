\documentclass[12pt]{extreport} % Schriftgröße: 8pt, 9pt, 10pt, 11pt, 12pt, 14pt, 17pt oder 20pt

%% Packages
\usepackage{scrextend}
\usepackage{amssymb}
\usepackage{amsthm}
\usepackage{booktabs}
\usepackage{chngcntr}
\usepackage{cmap}
\usepackage{color}
\usepackage{enumitem}
\usepackage{float}
\usepackage{hyperref}
\usepackage{ulem}
\usepackage{lmodern}
\usepackage{makeidx}
\usepackage{mathtools}
\usepackage{xpatch}
\usepackage{pgfplots}
\pgfplotsset{compat=1.7}
\usetikzlibrary{calc}	
\usetikzlibrary{matrix}	

% Language Setup (Deutsch)
\usepackage[utf8]{inputenc} 
\usepackage[T1]{fontenc} 
\usepackage[ngerman]{babel}

\usepackage{csquotes}
% Options
\makeatletter%%  
  % Linkfarbe, {0,0.35,0.35} für Türkis, {0,0,0} für Schwarz, {1,0,0} für Rot, {0,0,0.85} für Blau
  \definecolor{linkcolor}{rgb}{0,0.35,0.35}
  % Zeilenabstand für bessere Leserlichkeit
  \def\mystretch{1.2} 
  % Publisher definieren
  \newcommand\publishers[1]{\newcommand\@publishers{#1}} 
  % Enumerate im 1. Level: \alph für a), b), ...
  \renewcommand{\labelenumi}{\alph{enumi})} 
  % Enumerate im 2. Level: \roman für (i), (ii), ...
  \renewcommand{\labelenumii}{(\roman{enumii})}
  % Zeileneinrückung am Anfang des Absatzes
  \setlength{\parindent}{0pt} 
  % Verweise auf Enumerate, z.B.: 3.2 a)
  \setlist[enumerate,1]{ref={\thesatz ~ \alph*)}}
  % Für das Proof-Environment: 'Beweis:' anstatt 'Beweis.'
  \xpatchcmd{\proof}{\@addpunct{.}}{\@addpunct{:}}{}{} 
  % Nummerierung der Bilder, z.B.: Abbildung 4.1
  \@ifundefined{thechapter}{}{\def\thefigure{\thechapter.\arabic{figure}}} 
  % Chapter-Nummerierung beginnen bei (0):
  \setcounter{chapter}{0}
\makeatother%

% Meta Setup (Für Titelblatt und Metadaten im PDF)
\title{Globale Optimierung}
\author{Prof. Dr. Oliver Stein}
\date{Sommersemester 2017}
\publishers{Karlsruher Institut für Technologie}

%% Math. Definitionen
\newcommand{\C}{\mathbb{C}}
\newcommand{\N}{\mathbb{N}}
\newcommand{\Q}{\mathbb{Q}}
\newcommand{\R}{\mathbb{R}}
\newcommand{\Z}{\mathbb{Z}}

%% Theorems (unnamedtheorem = Theorem ohne Namen)
\newtheoremstyle{named}{}{}{\normalfont}{}{\bfseries}{:}{0.25em}{#2 \thmnote{#3}}
\newtheoremstyle{nnamed}{}{}{\normalfont}{}{\bfseries}{:}{0.25em}{\thmnote{#3}}
\newtheoremstyle{itshape}{}{}{\itshape}{}{\bfseries}{:}{ }{}
\newtheoremstyle{normal}{}{}{\normalfont}{}{\bfseries}{:}{ }{}
\renewcommand*{\qed}{\hfill\ensuremath{\square}}

\theoremstyle{named}
\newtheorem{unnamedtheorem}{Theorem} \counterwithin{unnamedtheorem}{chapter}
\theoremstyle{nnamed}
\newtheorem*{unnamedtheorem*}{Theorem} 

\theoremstyle{itshape}
\newtheorem{satz}[unnamedtheorem]{Satz} 
\newtheorem*{definition}{Definition}
\newtheorem{hilfssatz}[unnamedtheorem]{Hilfssatz}
\newtheorem*{hilfssatz*}{Hilfssatz}

\theoremstyle{normal}
\newtheorem{beispiel}[unnamedtheorem]{Beispiel}
\newtheorem{folgerung}[unnamedtheorem]{Folgerung}
%\newtheorem{hilfssatz}[unnamedtheorem]{Hilfssatz}
\newtheorem{anwendung}[unnamedtheorem]{Anwendung}
\newtheorem{anwendungen}[unnamedtheorem]{Anwendungen}
\newtheorem*{anwendung*}{Anwendung}
\newtheorem*{beispiel*}{Beispiel}
\newtheorem*{beispiele}{Beispiele}
\newtheorem*{bemerkung}{Bemerkung} 
\newtheorem*{bemerkungen}{Bemerkungen}
\newtheorem*{bezeichnung}{Bezeichnung}
\newtheorem*{eigenschaften}{Eigenschaften}
\newtheorem*{folgerung*}{Folgerung}
\newtheorem*{folgerungen}{Folgerungen}
%\newtheorem*{hilfssatz*}{Hilfssatz}
\newtheorem*{regeln}{Regeln}
\newtheorem*{motivation}{Motivation}
\newtheorem*{erinnerung}{Erinnerung}
\newtheorem*{schreibweise}{Schreibweise}
\newtheorem*{schreibweisen}{Schreibweisen}
\newtheorem*{uebung}{Übung}
\newtheorem*{vereinbarung}{Vereinbarung}

%% Template
\makeatletter%
\DeclareUnicodeCharacter{00A0}{ } \pgfplotsset{compat=1.7} \hypersetup{colorlinks,breaklinks, urlcolor=linkcolor, linkcolor=linkcolor, pdftitle=\@title, pdfauthor=\@author, pdfsubject=\@title, pdfcreator=\@publishers}\DeclareOption*{\PassOptionsToClass{\CurrentOption}{report}} \ProcessOptions \def\baselinestretch{\mystretch} \setlength{\oddsidemargin}{0.125in} \setlength{\evensidemargin}{0.125in} \setlength{\topmargin}{0.5in} \setlength{\textwidth}{6.25in} \setlength{\textheight}{8in} \addtolength{\topmargin}{-\headheight} \addtolength{\topmargin}{-\headsep} \def\pulldownheader{ \addtolength{\topmargin}{\headheight} \addtolength{\topmargin}{\headsep} \addtolength{\textheight}{-\headheight} \addtolength{\textheight}{-\headsep} } \def\pullupfooter{ \addtolength{\textheight}{-\footskip} } \def\ps@headings{\let\@mkboth\markboth \def\@oddfoot{} \def\@evenfoot{} \def\@oddhead{\hbox {}\sl \rightmark \hfil \rm\thepage} \def\chaptermark##1{\markright {\uppercase{\ifnum \c@secnumdepth >\m@ne \@chapapp\ \thechapter. \ \fi ##1}}} \pulldownheader } \def\ps@myheadings{\let\@mkboth\@gobbletwo \def\@oddfoot{} \def\@evenfoot{} \def\sectionmark##1{} \def\subsectionmark##1{}  \def\@evenhead{\rm \thepage\hfil\sl\leftmark\hbox {}} \def\@oddhead{\hbox{}\sl\rightmark \hfil \rm\thepage} \pulldownheader }	\def\chapter{\cleardoublepage  \thispagestyle{plain} \global\@topnum\z@ \@afterindentfalse \secdef\@chapter\@schapter} \def\@makeschapterhead#1{ {\parindent \z@ \raggedright \normalfont \interlinepenalty\@M \Huge \bfseries  #1\par\nobreak \vskip 40\p@ }} \newcommand{\indexsection}{chapter} \patchcmd{\@makechapterhead}{\vspace*{50\p@}}{}{}{}\def\Xint#1{\mathchoice
    {\XXint\displaystyle\textstyle{#1}} {\XXint\textstyle\scriptstyle{#1}} {\XXint\scriptstyle\scriptscriptstyle{#1}} {\XXint\scriptscriptstyle\scriptscriptstyle{#1}} \!\int} \def\XXint#1#2#3{{\setbox0=\hbox{$#1{#2#3}{\int}$} \vcenter{\hbox{$#2#3$}}\kern-.5\wd0}} \def\dashint{\Xint-} \def\Yint#1{\mathchoice {\YYint\displaystyle\textstyle{#1}} {\YYYint\textstyle\scriptscriptstyle{#1}} {}{} \!\int} \def\YYint#1#2#3{{\setbox0=\hbox{$#1{#2#3}{\int}$} \lower1ex\hbox{$#2#3$}\kern-.46\wd0}} \def\YYYint#1#2#3{{\setbox0=\hbox{$#1{#2#3}{\int}$}  \lower0.35ex\hbox{$#2#3$}\kern-.48\wd0}} \def\lowdashint{\Yint-} \def\Zint#1{\mathchoice {\ZZint\displaystyle\textstyle{#1}}{\ZZZint\textstyle\scriptscriptstyle{#1}} {}{} \!\int} \def\ZZint#1#2#3{{\setbox0=\hbox{$#1{#2#3}{\int}$}\raise1.15ex\hbox{$#2#3$}\kern-.57\wd0}} \def\ZZZint#1#2#3{{\setbox0=\hbox{$#1{#2#3}{\int}$} \raise0.85ex\hbox{$#2#3$}\kern-.53\wd0}} \def\highdashint{\Zint-} \DeclareRobustCommand*{\onlyattoc}[1]{} \newcommand*{\activateonlyattoc}{ \DeclareRobustCommand*{\onlyattoc}[1]{##1} } \AtBeginDocument{\addtocontents{toc} {\protect\activateonlyattoc}} 
	% Titlepage
	\def\maketitle{ \begin{titlepage} 
			~\vspace{3cm} 
		\begin{center} {\Huge \@title} \end{center} 
		\begin{center} {\LARGE Übung }\end{center}
	 		\vspace*{1cm} 
	 	\begin{center} {\large \@author} \end{center} 
	 	\begin{center} \@date \end{center} 
	 		\vspace*{7cm} 
	 	\begin{center} \@publishers \end{center} 
	 		\vfill 
	\end{titlepage} }
\makeatother%

% Indexdatei erstellen
\makeindex 

\begin{document}

\pagenumbering{Alph}
\begin{titlepage}
	\maketitle
	\thispagestyle{empty}
\end{titlepage}

% Skript - Anfang 			
\pagenumbering{arabic}
	
\subsection*{Aufgabe 1.1.}
Es sei $V$ ein Vektorraum über $\R$. Eine Abbildung $\| \cdot \| \cdot V \rightarrow [0, \infty)$ heißt Norm, wenn für alle $x, y \in V $ und $\alpha \in \R$ die folgenden Bedingungen erfüllt sind:
\begin{enumerate}[label=\arabic*\upshape)]
	\item Definitheit: $\| x \| = 0 \Rightarrow x = 0$.
	\item Absolute Homogenität: $\| \alpha x \| = |\alpha| \cdot \| x \|$
	\item Dreiecksungleichung: $\| x + y\| \leq \| x \| + \| y \|$
\end{enumerate}
Es sei $n \in \N$.
\begin{enumerate}
	\item Zeigen Sie dass für jede Norm in Bedingung 1) auch die Rückrichtung gilt
		\begin{proof}
			Zu zeigen: $\| 0 \| = 0$, wobei zu beachten ist, dass die erste 0 ein Element im Vektorraum $V$ darstellt und die zweite einen Skalar im zugrunde liegendem Raum. ~\\

			Es gilt für alle $x \in V$:
			$$\| 0 \| = \| 0 \cdot x \| = 0 \cdot \| x \| = 0 $$
		\end{proof}
	\item Zeigen Sie, dass die folgenden Abbildungen Normen auf $\R^n$ sind:
		\begin{enumerate}
			\item $\| \cdot \|_{1} \colon \R^n \rightarrow \R, ~ x \mapsto \sum_{i = 1}^{n} |x_i|$
				\begin{proof} ~\
					\begin{itemize}
						\item Nicht-Negativität: $\sum_{i=1}^{n} \underbrace{|x_i|}_{\geq 0} ~\Rightarrow ~\sum_{i=1}^{n} |x_i| \geq 0$
						\item Definitheit: $\| x \|_1 = \sum_{i=1}^{n} |x_i| \xRightarrow[]{|\cdot| \geq 0} |x_i| = 0 \quad \forall i = 1, \dotsc, n$
								$$ \qquad \xRightarrow[]{Definitheit} x_i = 0 \quad \forall i = 1, \dotsc, n $$
						\item Absolute Homogenität: $	\| \alpha x \|_1 = \sum_{i=1}^{n} |  \alpha x_i|  = \sum_{i=1}^{n} |\alpha| |x_i|$
							$$= |\alpha| \sum_{i=1}^{n} |x_i|  = |\alpha| \| x \|_1 $$
						\item Dreieckunsgleichung:	$\| x + y\|_1    = \sum_{i=1}^{n} | x_i   +   y_i |  \leq \sum_{i=1}^{n} \left( | x_i | + | y_i | \right)$
									$$\qquad  = \sum_{i=1}^{n} | x_i | + \leq \sum_{i=1}^{n} | y_i |   = \| x \|_1 + \| y \|_1 $$	
					\end{itemize}
				\end{proof}
			\item  $\| \cdot \|_{2} \colon \R^n \rightarrow \R, ~ x \mapsto \sqrt{\sum_{i = 1}^{n} |x_i|^2}$
				\begin{proof} ~\
					\begin{itemize}
						\item Nicht-Negativität: Wir definieren hierzu:
							\begin{align*}
								& f \colon \R^n \rightarrow [0, \infty)^n, ~(x_1, \dotsc, x_n) \mapsto (x_1^2, \dotsc, x_n^2) \\
								& g \colon [0, \infty)^n \mapsto [0, \infty), ~ y \mapsto \sum_{i=1}^{n} y_i \\
								& h \colon [0, \infty) \mapsto [0, \infty) , ~ z \mapsto \sqrt{z}
							\end{align*}
							Somit ist: $\|\cdot \|_2 = \left( h \circ g \circ f \right) \colon \R^n \rightarrow [0 \infty)$ 
						\item Definitheit: $0 = \|x\|_2 \iff \| x \|_2^2 = 0 \iff \sum_{i=1}^{n} x_i^2 = 0$, wobei $x_i^2 \geq 0$
						$$ \Rightarrow x_i = 0 ~\forall i \in \{1, \dotsc, n \} \iff x = 0$$
						\item Absolute Homogenität: Analog zu $\| \cdot \|_1$.
						\item Dreiecksungleichung: $\| x + y\|^2_2 = \left( x + y \right)^T \left( x + y \right) = x^T x  x^Ty + y^T x + y^T y $
						\begin{align*} 
							& = \| x \|^2_2 + 2 x^T y + \| y \|^2_2 \\
							& \leq \| x \|^2_2 + 2 | x^T y | + \| y \|^2_2 \\
							& \leq \| x \|^2_2 + 2 \| x\|_2 \|y \|_2 + \| y \|^2_2 \\
							& = \left( \|x\|_2 + \| y \|_2 \right)^2
						\end{align*}
							
					\end{itemize}
				\end{proof}
			\item $\| \cdot \|_{\infty} \colon \R^n \rightarrow \R, ~x \mapsto \max_{i =1}^{n} |x_i|$
				\begin{proof} ~\
					\begin{itemize}
						\item Nicht-Nevativität: Analog zu $\| \cdot \|_1$ oder $\| \cdot \|_2$
						\item Definitheit: $0 = \| x \|_{\infty}$$\iff\max_{i = 1}^{n} |x_i| = 0$$\iff|x_i| = 0 ~\forall i \in \{1, \dotsc, n \}$
							$$ \iff x_i = 0 ~\forall i \in \{1, \dotsc, n \} \iff x = 0. $$
						\item Absolute Homogenität: Analog zu $\| \cdot \|_1$ oder $\| \cdot \|_2$
						\item Dreiecksungleichung: $\| x + y\|_{\infty} = \max_{i=1}^{n} \left( \left| x_i + y_i \right| \right) = \max_{i=1}^{n} \left( |x_i| + |y_i| \right) $
							$$ \leq \max_{i = 1, \dotsc, n} \left( \max_{j=1}^{n} |x_j| + \max_{k=1}^{n} |y_k| \right) \leq \max_{j=1}^{n} |x_j| + \max_{k=1}^{n} |y_k|  = \| x \|_{\infty} + \| y \|_{\infty} $$ 
					\end{itemize}
				\end{proof}
		\end{enumerate}
\end{enumerate} ~\newline

\subsection*{Aufgabe 1.2.}

\begin{figure*}[h!] \centering
	\begin{tikzpicture}
		\begin{axis}[colormap/cool, scale=0.9]
			\addplot3[mesh, samples=50, domain=-1:1]{abs(x)+abs(y))};
			\addlegendentry{$\|x+y\|_{1}$}
		\end{axis}
	\end{tikzpicture}
\end{figure*}

\begin{figure*}[h!] \centering
	\begin{tikzpicture}
		\begin{axis}[colormap/cool, scale=0.9]
			\addplot3[mesh, samples=50, domain=-1:1]{x^2+y^2)};
			\addlegendentry{$\|x+y\|_{2}$}
		\end{axis}
	\end{tikzpicture}
\end{figure*}

\begin{figure*}[h!] \centering
	\begin{tikzpicture}
		\begin{axis}[colormap/cool, scale=0.9]
			\addplot3[mesh, samples=50, domain=-1:1]{max(abs(x),abs(y))};
			\addlegendentry{$\|x+y\|_{\infty}$}
		\end{axis}
	\end{tikzpicture}
\end{figure*}

\subsection*{Aufgabe 1.3.}

Gegeben seien eine Menge von zulässigen punkten $M \subseteq \R^n$ und eine Zielfunktion $f \colon M \rightarrow \R$. Zeigen Sie: 

\begin{enumerate}
	\item Die globalen Maximalpunkte von $f$ auf $M$ sind genau die globalen Minimalpunkte von $-f$ auf $M$.
		\begin{proof}
			Es gilt: $x^*$ ist globaler Maximalpunkt von $f$ auf $M$:
			\begin{align*}
				& \iff f(x^*) \geq f(x) \quad \forall x \in M \\ 
				& \iff - f(x^*) \leq -f(x) \quad \forall x \in M \\
				& \iff x^* \text{ ist globaler Minimalpunkt von } -f \text{ auf } M
			\end{align*}
		\end{proof}
	\item Sofern $f$ globale Maximalpunkte besitzt, gilt für den globalen Maximalwert
	$$ \max_{x \in M} f(x) = - \min_{x \in M} \left( - f(x) \right). $$
		\begin{proof}
		todo
		\end{proof}
\end{enumerate}

\newpage

\subsection*{Aufgabe 2.1}

\begin{enumerate}
	\item Es seien $n \in \N$, $g \colon \R^n \rightarrow \R$ eine stetige Abbildung und $M \coloneqq \{ x \in R^{n} | (x) \leq 0 \}$. Zeigen Sie, dass $M$ abgeschlossen ist.
		\begin{proof}
			$M = \{ x \in R^n | g(x) \leq 0 \}$, sei $(x_n)_n \subseteq M$, $\lim x_n = x^*$. Zu zeigen ist $x^* \in M$ bzw. $g(x^*) \leq 0$. Wir nutzen hierfür die Stetigkeit von $g$ aus, denn damit ist:
			
			$$ g(x^*) = g(\lim x_n ) = \lim \underbrace{g(x_n)}_{\leq 0} \leq 0 $$
		\end{proof}
	\item Zeigen Sie, dass die Menge $M$ aus Aufgabenteil a) nicht beschränkt sein muss.
		\begin{proof}
			Als Gegenbeispiel sei $g(x) = -x^2$. Damit ist $M = \R$, was nicht beschränkt ist.
		\end{proof}
	\item Es seien $I$ eine beliebige Indexmenge, $n \in \N$, $g_i \colon \R^n \rightarrow \R$, $i \in I$, stetige Abbildungen und $M_i \coloneqq \{ x \in \R^n | g_i(x) \leq 0 \}$. Zeigen Sie, dass 
		$$ M \coloneqq \bigcap_{i \in I} M_i $$
		abgeschlossen ist.
		\begin{proof}
			Sei $(x_n)_n \subseteq M$ mit $\lim x_n = x*^*$. ZU zeigen ist $x^* \in M$. Da 
				$$(x_n)_n \subseteq M= \bigcap M_i$$
				folgt $(x_n)_n \subseteq M_i$ für alle $i \in I$. Da $M_i$ für alle $i \in I$ abgeschlossen ist, folgt somit:
			$$ x^* \in M_i ~\forall i \in I \iff x^* \in \bigcap_{i \in I} M_i \iff x^* \in M $$
		\end{proof}
\end{enumerate}


\subsection*{Aufgabe 2.2}

Zeigen oder widerlegen Sie die Koerzivität der folgenden Funktionen auf $\R^2$

\begin{enumerate}
	\item $f(x_1, x_2) = x_1^3 - x_2$,
		\begin{proof}
			Sei $(x_n)_n = (0, \nu)$. Damit ist $\| x_\nu \| = \nu \xrightarrow[\nu \rightarrow \infty]{} \infty$, allerdings
			$$ f(x_n) = - \nu \rightarrow - \infty $$
			für $\nu \rightarrow \infty$. Ein anderes Beispiel wäre $(y_n)_n = (n, n^3)$, denn damit ist $\| y_n \| \rightarrow \infty$ und $f(y_n) = 0$ für alle $n \in \N$.
		\end{proof}
	\item $f(x_1, x_2) = \left(x_1^4 + x_2^4 \right) e^{x_1^2 + x_2^2}$
		\begin{proof}
			Wir können die Funktion nach unten hin abschätzen:
			$$ f(x) = \left( x_1^4 + x_2^4 \right) \underbrace{e^{x_1^2 + x_2^2}}_{\geq 1} \geq \left( x_1^4 + x_2^4 \right) = \| x \|_4^4 $$
			Damit ist für $(x_n)_n \subseteq \R^2$ mit $\| x \|_4 \rightarrow \infty$:
			$$ \lim_{n \rightarrow \infty} f(x_n) \geq \lim_{n \rightarrow \infty} \| x_n \|_4^4 \rightarrow \infty $$
		\end{proof}
\end{enumerate}
bzw. der folgenden Funktion auf $\R^n$
\begin{enumerate}   \setcounter{enumi}{2}
	\item $f(x) = x^T A x$, mit $n \in \N$ und $A \in \R^{n\times n}$ positiv definit aber \textit{nicht notwendigerweise symmetrisch}.
		\begin{proof}
			Es ist $f(x) = \langle x, Ax \rangle$ und damit
			\begin{align*}
				f(x) & = \frac{1}{2} \langle x, Ax \rangle + \frac{1}{2} \langle   x, A x \rangle \\
					 & = \frac{1}{2} \langle x, Ax \rangle + \frac{1}{2} \langle A x, x   \rangle \\
					 & = \frac{1}{2} x^T A x + \frac{1}{2} x^T A^T x \\
					 & = x^T \underbrace{\left( \frac{A + A^T}{2} \right)}_{\eqqcolon \tilde{A}} x
			\end{align*}
			Da aber $\tilde{A}$ symmetrisch ist per Konstruktion, so folgt mit dem Hinweis die Behauptung da:
			$$ f(x) = x^T \tilde{A} x = \| x \|_{\tilde{A}}^2 $$
		\end{proof}
\end{enumerate}

\textbf{Hinweis}: Eine positiv definite \textit{und symmetrische Matrix} $A \in \R^{n \times n}$ induziert mit $\langle x y \rangle_A \coloneqq x^TAy$ ein Skalarprodukt, welches wiederum durch $\| x \|_A \coloneqq \sqrt{\langle x, x \rangle_A}$ eine Norm induziert.


\subsection*{Aufgabe 2.3}

Sei $p \colon \R^2 \rightarrow \R$ definiert durch
$$ p(x) = - \frac{1 + 4 x^2_1}{1 + 2 x_1^2} + 2 x_2^2 $$
Bestimmen Sie das Infimum von $p$ auf der Menge $D = \{ x \in \R^2 | x_1 \geq 1, x_2 > 0 \}$ und zeigen Sie, dass das Infimum nicht als Minimalwert angenommen wird.

\begin{proof}
	Zu zeigen: $\forall x \in D: p(x) \geq -2$, $\inf_{x \in D} f(x) = - 2$, $\not\exists x \in D: f(x) = -2$
	
	\begin{align*}
		f(x) & = - \frac{1 + 4x_1^2}{1 + 2x_1^2} + 2 x_2^2 \\
			& = - \frac{2 + 4 x_1^2 - 1}{1 + 2x_1^2	} + 2 x_2^2 \\
			& = - \frac{2 ( 1+ 2 x_1^2) - 1}{1 + 2 x_1^2} + 2x_2^2 \\
			& = - 2 + \underbrace{\frac{1}{1 + 2 x_1^2}}_{\geq 0} + \underbrace{2 x_2^2}_{> 0} \geq -2
	\end{align*}
 	Wählen wir nun $(x_n)_n \subseteq M$ mit $x_n = \left( n , \frac{1}{n} \right)$
 	$$ f(x_n) = - 2  \frac{1}{1 + 2 n ^2} + 2 \frac{1}{n^2} \rightarrow  $$
\end{proof}

% Skript - Ende 			

\end{document}