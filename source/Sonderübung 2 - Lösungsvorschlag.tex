\documentclass[12pt]{extreport} % Schriftgröße: 8pt, 9pt, 10pt, 11pt, 12pt, 14pt, 17pt oder 20pt

%% Packages
\usepackage{scrextend}
\usepackage{amssymb}
\usepackage{amsthm}
\usepackage{booktabs}
\usepackage{caption}
\usepackage{subcaption}
\usepackage{chngcntr}
\usepackage{cmap}
\usepackage{color}
\usepackage{csquotes}
\usepackage{enumitem}
\usepackage{float}
\usepackage{graphicx}
\usepackage{hyperref}
\usepackage{ulem}
\usepackage{lmodern}
\usepackage{makeidx}
\usepackage{amsmath}
\usepackage{mathtools}
\usepackage{xpatch}
\usepackage{pgfplots}
\pgfplotsset{compat=1.12}
\usepgfplotslibrary{fillbetween}
\usepackage{amsfonts}
\usetikzlibrary{calc}	
\usetikzlibrary{matrix}	
\usepackage{fancyhdr}
\usepackage{epstopdf}



% Language Setup (English)
\usepackage[utf8]{inputenc} 
\usepackage[T1]{fontenc} 
\usepackage[ngerman]{babel}

% Options
\makeatletter%%  
  % Linkfarbe, {0,0.35,0.35} für Türkis, {0,0,0} für Schwarz, {1,0,0} für Rot, {0,0,0.85} für Blau
  \definecolor{linkcolor}{rgb}{0,0.35,0.35}
  % Zeilenabstand für bessere Leserlichkeit
  \def\mystretch{1.2} 
  % Publisher definieren
  \newcommand\publishers[1]{\newcommand\@publishers{#1}} 
  % Enumerate im 1. Level: \alph für a), b), ...
  \renewcommand{\labelenumi}{\alph{enumi})} 
  % Enumerate im 2. Level: \roman für (i), (ii), ...
  \renewcommand{\labelenumii}{(\roman{enumii})}
  % Zeileneinrückung am Anfang des Absatzes
  \setlength{\parindent}{0pt} 
  % Für das Proof-Environment: 'Beweis:' anstatt 'Beweis.'
  \xpatchcmd{\proof}{\@addpunct{.}}{\@addpunct{:}}{}{} 
  % Nummerierung der Bilder, z.B.: Abbildung 4.1
  \@ifundefined{thechapter}{}{\def\thefigure{\thechapter.\arabic{figure}}} 
  % Chapter-Nummerierung beginnen bei (0):
  \setcounter{chapter}{0}
  % Chapter-Nummerierung
  \renewcommand\thechapter{\Roman{chapter}}
\makeatother%

% Meta Setup 
\title{Globale Optimierung - Sonderübung I}
\author{Kostorz, Belica}
\date{Sommersemester 2017}
\publishers{Karlsruher Institut für Technologie}

%% Math. Definitiones
\newcommand{\C}{\mathbb{C}}
\newcommand{\N}{\mathbb{N}}
\newcommand{\Q}{\mathbb{Q}}
\newcommand{\R}{\mathbb{R}}
\newcommand{\Z}{\mathbb{Z}}
\newcommand{\DO}[1]{\mathcal{D}\left( {#1} \right)}
\newcommand{\RO}[1]{\mathcal{R}\left( {#1} \right)}

\newtheoremstyle{named}{}{}{\normalfont}{}{\bfseries}{:}{0.25em}{#2 \thmnote{#3}}
\newtheoremstyle{nnamed}{}{}{\normalfont}{}{\bfseries}{:}{0.25em}{\thmnote{#3}}
\newtheoremstyle{itshape}{}{}{\itshape}{}{\bfseries}{:}{ }{}
\newtheoremstyle{normal}{}{}{\normalfont}{}{\bfseries}{:}{ }{}
\renewcommand*{\qed}{\hfill\ensuremath{\square}}

\theoremstyle{named}
\newtheorem{unnamedtheorem}{Theorem} \counterwithin{unnamedtheorem}{chapter}
\theoremstyle{nnamed}
\newtheorem*{unnamedtheorem*}{Theorem} 

\theoremstyle{itshape}
\newtheorem{definition}[unnamedtheorem]{Definition}

\theoremstyle{normal}
\newtheorem*{recall}{Recall}
\newtheorem*{example}{Example}
\newtheorem*{theorem}{Theorem}
\newtheorem*{remark}{Remark}
\newtheorem*{satz}{Satz}
\newtheorem*{bemerkung}{Bemerkung}



\fancypagestyle{firststyle}
{
   \fancyhf[C]{\small Globale Optimierung - Sonderübung II - Nadine Kostorz (1972005), Martin Belica (1775706)}
   \fancyfoot[C]{}
}
%% Template
\makeatletter%
\DeclareUnicodeCharacter{00A0}{ } \pgfplotsset{compat=1.7} \hypersetup{colorlinks,breaklinks, urlcolor=linkcolor, linkcolor=linkcolor, pdftitle=\@title, pdfauthor=\@author, pdfsubject=\@title, pdfcreator=\@publishers}\DeclareOption*{\PassOptionsToClass{\CurrentOption}{report}} \ProcessOptions \def\baselinestretch{\mystretch} \setlength{\oddsidemargin}{0.125in} \setlength{\evensidemargin}{0.125in} \setlength{\topmargin}{0.5in} \setlength{\textwidth}{6.25in} \setlength{\textheight}{8in} \addtolength{\topmargin}{-\headheight} \addtolength{\topmargin}{-\headsep} \def\pulldownheader{ \addtolength{\topmargin}{\headheight} \addtolength{\topmargin}{\headsep} \addtolength{\textheight}{-\headheight} \addtolength{\textheight}{-\headsep} } \def\pullupfooter{ \addtolength{\textheight}{-\footskip} } \def\ps@headings{\let\@mkboth\markboth \def\@oddfoot{} \def\@evenfoot{} \def\@oddhead{\hbox {}\sl \rightmark \hfil \rm\thepage} \def\chaptermark##1{\markright {\uppercase{\ifnum \c@secnumdepth >\m@ne \@chapapp\ \thechapter. \ \fi ##1}}} \pulldownheader } \def\ps@myheadings{\let\@mkboth\@gobbletwo \def\@oddfoot{} \def\@evenfoot{} \def\sectionmark##1{} \def\subsectionmark##1{}  \def\@evenhead{\rm \thepage\hfil\sl\leftmark\hbox {}} \def\@oddhead{\hbox{}\sl\rightmark \hfil \rm\thepage} \pulldownheader }	\def\chapter{\cleardoublepage  \thispagestyle{plain} \global\@topnum\z@ \@afterindentfalse \secdef\@chapter\@schapter} \def\@makeschapterhead#1{ {\parindent \z@ \raggedright \normalfont \interlinepenalty\@M \Huge \bfseries  #1\par\nobreak \vskip 40\p@ }} \newcommand{\indexsection}{chapter} \patchcmd{\@makechapterhead}{\vspace*{50\p@}}{}{}{}\def\Xint#1{\mathchoice
    {\XXint\displaystyle\textstyle{#1}} {\XXint\textstyle\scriptstyle{#1}} {\XXint\scriptstyle\scriptscriptstyle{#1}} {\XXint\scriptscriptstyle\scriptscriptstyle{#1}} \!\int} \def\XXint#1#2#3{{\setbox0=\hbox{$#1{#2#3}{\int}$} \vcenter{\hbox{$#2#3$}}\kern-.5\wd0}} \def\dashint{\Xint-} \def\Yint#1{\mathchoice {\YYint\displaystyle\textstyle{#1}} {\YYYint\textstyle\scriptscriptstyle{#1}} {}{} \!\int} \def\YYint#1#2#3{{\setbox0=\hbox{$#1{#2#3}{\int}$} \lower1ex\hbox{$#2#3$}\kern-.46\wd0}} \def\YYYint#1#2#3{{\setbox0=\hbox{$#1{#2#3}{\int}$}  \lower0.35ex\hbox{$#2#3$}\kern-.48\wd0}} \def\lowdashint{\Yint-} \def\Zint#1{\mathchoice {\ZZint\displaystyle\textstyle{#1}}{\ZZZint\textstyle\scriptscriptstyle{#1}} {}{} \!\int} \def\ZZint#1#2#3{{\setbox0=\hbox{$#1{#2#3}{\int}$}\raise1.15ex\hbox{$#2#3$}\kern-.57\wd0}} \def\ZZZint#1#2#3{{\setbox0=\hbox{$#1{#2#3}{\int}$} \raise0.85ex\hbox{$#2#3$}\kern-.53\wd0}} \def\highdashint{\Zint-} \DeclareRobustCommand*{\onlyattoc}[1]{} \newcommand*{\activateonlyattoc}{ \DeclareRobustCommand*{\onlyattoc}[1]{##1} } \AtBeginDocument{\addtocontents{toc} {\protect\activateonlyattoc}} \newcommand{\RightArrow}{\xRightarrow[]{ ~ ~ }} \newcommand{\LeftArrow}{\xLeftarrow[]{ ~ ~ }} \newcommand{\rightArrow}{\xrightarrow[]{ ~ ~ }} \newcommand{\leftArrow}{\xleftarrow[]{ ~ ~ }}
	% Titlepage
	\def\maketitle{ \begin{titlepage} 
			~\vspace{3cm} 
		\begin{center} {\Huge \@title} \end{center} 
	 		\vspace*{1cm} 
	 	\begin{center} {\large \@author} \end{center} 
	 	\vspace*{-0.5cm}
	 	\begin{center} \@date \end{center} 
	 		\vspace*{7cm} 
	 	\begin{center} \@publishers \end{center} 
	 		\vfill 
	\end{titlepage} }
\makeatother%

% Create Index
\makeindex 

\begin{document}

\thispagestyle{empty}


\pagenumbering{arabic}\thispagestyle{firststyle}

\subsubsection{Aufgabe S. 2.1 (\textit{$2 + 2 + 3 + 3$})} 

Gegeben sei das Optimierungsproblem
$$ P: \quad \min_{w \in \R^m} \frac{1}{2} \| w \|_2^2 \text{ s.t. } Aw \geq e $$
mit einer $(d, m)$-Matrix $A$ und
$$ e \coloneqq \left(\begin{array}{c} 1 \\ \vdots \\ 1 \end{array}\right) \in \R^d. $$
Gehen Sie im Folgenden davon aus, dass ein $\tilde{w}$ existiert, für das gilt
$$ A \tilde{w} \geq e. $$
\begin{enumerate}
	\item Zeigen Sie, dass $P$ lösbar ist.
		\begin{proof}
			Nach Korollar 1.2.40 gilt:
				$$ \textit{Es sei } M \textit{ nicht-leer, und } f \colon M \rightarrow R \textit{ sei stetig und koerziv auf } M. $$
				$$ \textit{ Dann ist } S \textit{ nicht-leer und kompakt.} $$ ~\\
			Nach Aufgabenstellung ist mit $f(w) = \frac{1}{2} \| w \|_2^2$ und $g(w) = e - Aw$, sodass 
				$$ M = \{ w \in \R^m \colon  g(w) \leq 0 \} $$ 
			und damit:
				$$ P: \quad \min_{w \in \R^m} f(w) \text{ s.t. } w \in M. $$
			Nach Voraussetzung existiert ein $\tilde{w} \in M$, sodass $M$ nicht-leer ist. Weiter ist die Funktion $f$ stetig, da für $\| x - y \|_2 < \delta$ und $\epsilon \coloneqq \frac{\delta^2}{2}$ mit der umgekehrten Dreiecksungleichung gilt:	
			$$ \left| f(x) - f(y) \right| = \frac{1}{2} \left| \| x \|^2_2 - \| y \|_2^2 \right| \leq \frac{1}{2} \| x - y \|_2^2 \leq \epsilon. $$
			$M$ ist aufgrund der Stetigkeit der Restriktionsfunktion als untere Niveaumenge abgeschlosse\footnote{Dies wurde in der Übung (o. B.) kurz angesprochen; deshalb ist ein kurzer Beweis hinten angehängt} bzw. falls man anstatt $g$ komponentenweise $g_i$ als Restriktion betrachtet, so ist $M$ als Schnitt abgeschlossener Mengen ebenfalls abgeschlossen:
			$$ M = \bigcap_{i \in I } \left\{ w \in \R^m \colon g_i(w) \leq 0 \right\}. $$
			 Aus der Abgeschlossenheit der Menge $M$ folgt, dass die Funktion $f$ koerziv ist, denn nach Definition 1.2.37 müssen nur alle Folgen $(x^\nu) \subseteq M$ mit $\| x^\nu \| \rightarrow \infty$ betrachtet werden und für die gilt mittels der Äquivalenz der Normen im $\R^m$:
			$$ f(x^\nu) = \frac{1}{2} \| x^\nu \|_2^2 \longrightarrow +\infty. $$
			Damit sind alle Voraussetzungen von Korollar 1.2.40 erfüllt, $S$ ist somit nicht-leer und kompakt und somit $P$ lösbar.
		\end{proof}
	\item Zeigen Sie, dass $P$ ein konvexes Optimierungsproblem ist.
		\begin{proof}
		Nach Beispiel 2.1.9 gilt: ~\\
			\textit{Falls $f$, $g_i$ für $i \in I$ auf $\R^m$ konvexe Funktionen sind, dann ist}
			$$P: \quad \min f(x) \text{ s.t. } g_i (x) \leq 0, ~ i \in I $$
			\textit{ein konvexes Optimierungsproblem.} ~\\
			
			Beim gegebenen Problem ist $g_i(w) = - (A w)_i + 1$ für alle $i \in \{1, \dotsc, d \}$ eine lineare und somit konvexe Funktion. Da $f \in C^2(\R^m)$ mit $f(w) = \frac{1}{2} \sum_{i=1,\dotsc, m} w_i^2$ folgt $\nabla_w f(w) = w$ und 
			$$ D^2 f =  \left(\begin{array}{rrrrr} 1 & 0 & \dotsc & 0 \\ 0 & 1 & 0 & \vdots \\ \vdots & 0 & \ddots & 0\\ 0 &  \dotsc & 0 & 1 \end{array}\right) \succ 0, $$
			womit nach 2.5.10 $f$ insbesondere konvex und $P$ ein konvexes Optimierungsproblem ist.
		\end{proof}
	\item Stellen Sei das Wolfe-Dual $D$ zu $P$ auf.
		\begin{proof}
			Die Lagrange Funktion zum restringierten, konvexen Optimierungsproblem $P$ lautet:
			\begin{align*}
				L(w, \lambda) = f(w) + \sum_{i \in I} \lambda_i g_i(w) & = \frac{1}{2} \| w \|_2^2 + \lambda^T \left( e - Aw \right) \\
					& =\frac{1}{2} \sum_{i=1}^m w_i^2 + \sum_{i=1}^d \lambda_i \left(1 - a_i \cdot w \right)
			\end{align*}
			mit $a_i$ i-ter Zeilenvektor von $A$ und $\lambda = \left( \lambda_1, \dotsc, \lambda_d \right)^T$. Das Dualproblem lässt sich dadurch folgendermaßen umformulieren:
			$$ D: \quad \max_{w,\lambda} L(w, \lambda) \text{ s.t. } \nabla_w L(w, \lambda) = 0, ~\lambda \geq 0, $$
			mit der zulässigen Menge
			$$ M_D = \left\{ (w, \lambda) \in \R^m \times \R^d \colon \nabla_w L(w, \lambda) = 0, \lambda \geq 0 \right\}. $$
			Dabei ist 
			\begin{align*}
			 \nabla_w L(w, \lambda) & = w - \nabla_w \left( \sum_{i=1,\dotsc,d} \lambda_i \sum_{k=1, \dotsc, m} A_{ik}w_{k} \right) \\
			& = w - \nabla_w \left( \sum_{k=1, \dotsc, m}  \sum_{i=1,\dotsc,d}  \lambda_i A_{ik}w_{k} \right) \\
			& = \left(\begin{array}{c} w_1 -\sum_{i=1,\dotsc,d}  \lambda_i A_{i1}  \\ \vdots \\ w_m -\sum_{i=1,\dotsc,d}  \lambda_i A_{im} \end{array}\right) 
			\end{align*}
			Zusammengefasst also:
			$$ D: \quad \max_{(w, \lambda) \in \R^m \times \R^d} \frac{1}{2} \sum_{i=1}^m w_i^2 + \sum_{i=1}^d \lambda_i \left(1 - a_i \cdot w \right) \text{ s.t. } \lambda \geq 0, $$
			$$ \text{ und } \left(\begin{array}{c} w_1 -\sum_{i=1,\dotsc,d}  \lambda_i A_{i1}  \\ \vdots \\ w_m -\sum_{i=1,\dotsc,d}  \lambda_i A_{im} \end{array}\right) = 0 $$
		\end{proof}
	\item Formulieren Sie, ausgehend von Aufgabenteil b), ein duales Optimierungsproblem, das nur von der Dualvariabel $\lambda$ abhängt.
		\begin{proof}
			Aus der Forderung aus c)
			$$ \left(\begin{array}{c} w_1 -\sum_{i=1,\dotsc,d}  \lambda_i A_{i1}  \\ \vdots \\ w_m -\sum_{i=1,\dotsc,d}  \lambda_i A_{im} \end{array}\right)  = 0 $$
			folgt $w_1 -\sum_{i=1,\dotsc,d}  \lambda_i A_{i1} = 0, ~ \dotsc, ~ w_m -\sum_{i=1,\dotsc,d}  \lambda_i A_{im} = 0$ und damit ist $$w_1 = \sum_{i=1,\dotsc,d}  \lambda_i A_{i1}, ~ \dotsc, ~ w_m = \sum_{i=1,\dotsc,d}  \lambda_i A_{im}. $$
			Somit lassen sich die $w_i$ in $D$ ersetzen und das Dualproblem reduziert sich zu
		 	$$ P_{red}: \quad \max_{\lambda \in \R^d} \sum_{i=1, \dotsc,m } \left( \sum_{j=1, \dotsc, d} \lambda_j A_{ji} \right)^2 + \sum_{i = 1, \dotsc, d} \lambda_i \left( 1 - \left(\sum_{k = 1, \dotsc, m} A_{ik}  \sum_{j=1, \dotsc, d} \lambda_j A_{jk} \right) \right) \text{ s.t. }\lambda \geq 0 $$
		 	$$ \iff \max_{\lambda \in \R^d}\| \lambda^T A \|_2^2 +  \lambda^T \left( e - A A^T \lambda  \right)  \text{ s.t. }\lambda \geq 0$$
		 	$$ \iff \max_{\lambda \in \R^d}\| \lambda^T A \|_2^2 +  \lambda^T e - \left( \lambda^T A \right) \left( \lambda^T A \right)^T  \text{ s.t. }\lambda \geq 0 $$
		\end{proof}
\end{enumerate}

\newpage

\subsubsection{Aufgabe S. 2.2 (\textit{$3 + 4 + 5$})}

Gegeben sei das Optimierungsproblem
$$ P: \quad \min_{x \in \R^2} x_1^2 + 2 x_2^2 + x_1 \text{ s.t. } x_1 + x_2 \leq \alpha $$
für einen Parameter $\alpha \in \R$.

\begin{enumerate}
	\item Zeigen Sie, ohne $P$ explizit zu lösen, dass $P$ einen \textbf{eindeutigen} globalen Minimalpunkt besitzt.
		\begin{proof}
			 $M$ ist trivialerweise nicht-leer, da die Menge nur durch eine lineare Funktion 
			 $$ g(x) = x_1+x_2-\alpha \leq 0 $$
			  beschränkt ist. Außerdem ist $M$ aufgrund der Stetigkeit (analog zu S. 2.1 a) der Restriktionsfunktion als untere Niveaumenge von $g$ abgeschlossen\footnote{vgl. S. 2.1 a) kurz bzw. kurzer Beweis im Anhang}. Da lineare Funktionen insbesondere konvex sind, ist nach Übung 4.4 die Menge $M$ konvex. Als Polynom ist $f \in C^2(\R^2)$ und Ableiten der Zielfunktion ergibt: \\
				$$ \nabla f(x) = \begin{pmatrix} 2x_1+1 \\ 4x_2 \end{pmatrix} \text{ und damit } D^2f(x) = \begin{pmatrix} 2 & 0 \\ 0 & 4 \end{pmatrix} \succ 0. $$
 			Nach 2.5.10 ist $f$ somit gleichmäßig konvex. Die eindeutige Lösbarkeit folgt nun mit Satz 2.3.3 c). 
		\end{proof}
	\item Lösen Sie $P$ mit Hilfe der KKT-Bedingungen ~\\
		(\textit{Hinweis: die Lösung ist abhängig von $\alpha$}).
		\begin{proof}
			Nach Definition ist 2.6.10 gilt:
			$$\textit{Für ein $C^1$-Problem $P$ heißt ein Punkt $x \in \R^n$ KKT-Punkt mit Multiplikatoren $\lambda$}$$
			$$\textit{und $\mu$ falls folgendes System von Gleichungen und Ungleichungen erfüllt ist:} $$
			$$ \nabla_x L(\overline{x}, \overline{\lambda}, \overline{\mu}) = 0, ~ \overline{\lambda}_i g_i(\overline{x}) = 0, ~\overline{\lambda}_i \geq 0, ~g_i(\overline{x}) \leq 0, ~ h_j(\overline{x}) = 0  $$
		
			für $i \in I$ und $j \in J$. Nach Satz 2.6.12 gilt außerdem:
			$$\textit{$P$ sei konvex und $C^1$, und $x$ sei KKT-Punkt (mit Multiplikatoren $\lambda$, $\mu$). }$$
			$$ \textit{Dann ist $x$ globaler Minimalpunkt von $P$.} $$
			
			Da $M$ nur durch eine Ungleichung beschrieben wird, lässt sich die Lagrange-Funktion schreiben als:
				$$L(x, \lambda) = x_1^2+2x_2^2+x_1+\lambda (x_1+x_2-\alpha). $$
			Damit ergibt sich für die erste KKT-Bedingung:
				$$ \nabla_x L(x, \lambda)=\begin{pmatrix} 2x_1+1+\lambda \\ 4x_2+ \lambda \end{pmatrix} =0. \Leftrightarrow  \begin{pmatrix}	x_1 \\ x_2\end{pmatrix} =  \begin{pmatrix} \frac{-1-\lambda}{2} \\ \frac{-\lambda}{4} \end{pmatrix}. $$
			Für die Komplementärbedinungen betrachte man die beiden folgenden Fälle. 
			\begin{description}
				\item[1. Fall] $ I_0(x)= \emptyset\Rightarrow \lambda_1=0$. Einsetzen von $\lambda_1=0$ in die 1. Bedingung ergibt 
					$$ \begin{pmatrix}	x_1 \\ x_2\end{pmatrix} =  \begin{pmatrix} \frac{-1}{2} \\ 0\end{pmatrix}. $$
				Nach Konstruktion werden die ersten 3 KKT-Bedninungen erfüllt und da $ I_0(x)= \emptyset $ gilt, muss zudem 
					$$x_1+x_2-\alpha < 0$$ 
					erfüllt sein. Einsetzen von $x_1, x_2$ ergibt $\frac{-1}{2} < \alpha$.
				\item[2. Fall] $ I_0(x)=\left\{1\right\} \Rightarrow x_1+x_2-\alpha= 0$. Einsetzen von $x_1$, $x_2$ ergibt: $$\frac{-1-\lambda}{2} + \frac{-\lambda}{4}-\alpha=0.$$
					$$ \iff \frac{-1}{2} -\frac{ 3}{4}\lambda=\alpha ~ \iff ~ \lambda= -\frac{4}{3}\alpha - \frac{2}{3} $$
					Da zudem $\lambda \geq 0$ gelten muss, ergibt sich $\alpha \leq - \frac{1}{2}$. Einsetzen von $\lambda$ in die 1. Bedingung ergibt 
			$$ \begin{pmatrix} x_1 \\ x_2\end{pmatrix} = \begin{pmatrix} \frac{-1}{2}+\frac{4}{6}\alpha+ \frac{2}{6} \\ \frac{4}{12}\alpha +\frac{2}{12}\end{pmatrix} = \begin{pmatrix} \frac{2}{3}\alpha- \frac{1}{6} \\ \frac{1}{3}\alpha +\frac{1}{6} \end{pmatrix}. $$
			d.h. $\lambda= -\frac{4}{3}\alpha - \frac{2}{3}$ mit $\begin{pmatrix} x_1 \\ x_2\end{pmatrix} = \begin{pmatrix} \frac{2}{3}\alpha- \frac{1}{6} \\ \frac{1}{3}\alpha +\frac{1}{6} \end{pmatrix}$ erfüllt nach Konstruktion alle KKT-Bedingungen. ~\\
			
			Unter Verwendung von Teilaufgabe a) und Satz 2.6.12 ist somit für $\alpha \leq - \frac{1}{2}$ 
			$$ \begin{pmatrix} x_1 \\ x_2\end{pmatrix} = \begin{pmatrix} \frac{2}{3}\alpha- \frac{1}{6} \\ \frac{1}{3}\alpha +\frac{1}{6} \end{pmatrix} 
			\text{ und für $\alpha > -\frac{1}{2}$ ist }
			\begin{pmatrix} x_1 \\ x_2\end{pmatrix} = \begin{pmatrix} -0.5 \\ 0 \end{pmatrix} $$
			globaler Minimalpunkt.
			\end{description}
		\end{proof}
	\item Es sei $v(\alpha)$ die Optimalwertfunktion von $P$ abhängig vom Parameter $\alpha$. Schreiben Sie $v$ explizit und zeigen Sie, dass $v$ eine konvexe Funktion ist.
		\begin{proof}
			Setzt man die obige Lösung für $\alpha \leq - 0.5$ nun in die Zielfunktion ein, so ergibt sich folgende von $\alpha$ abhängige Funktion für den Optimalwert von P:
			\begin{align*}
				v(\alpha) & = \left(\frac{2}{3}\alpha- \frac{1}{6} \right)^2 + 2 \cdot \left(\frac{1}{3}\alpha +\frac{1}{6} \right)^2+ \frac{2}{3} \alpha - \frac{1}{6} \\
				& = \frac{2}{3}\alpha^2+\frac{2}{3}\alpha - \frac{1}{12} \\
				& = \frac{2}{3} \left(\alpha^2 + \alpha \right) -\frac{1}{12} \\
				& = \frac{2}{3} \left(\alpha +\frac{1}{2} \right)^2-\frac{1}{4}.
			\end{align*}
			$v$ ist somit für $\alpha \leq - 0.5$ als Parabel insbesondere konvex, denn $v''(\alpha)=\frac{4}{3} > 0$. Für $\alpha > -0.5$ ist $v$ konstant und damit auch konvex. $v$ ist außerdem stetig, da 
			$$ \lim_{\alpha \rightarrow -0.5_{-}} v(\alpha) = \lim_{\alpha \rightarrow -0.5_{-}} \frac{2}{3} \left(\alpha +\frac{1}{2} \right)^2-\frac{1}{4} = - 0.25 = \left(-0.5 \right)^2 - 0.5 = \lim_{\alpha \rightarrow -0.5_{+}} v(\alpha). $$
			$v$ ist als Komposition stetig differenzierbarer Funktionen differenzierbar und es gilt
			$$ v'(\alpha) = \begin{cases} \frac{4}{3} \alpha + \frac{2}{3}, & \alpha \leq -0.5 \\ 0, & \alpha > -0.5\end{cases}, $$
			also 
			$$\lim_{\alpha \rightarrow -0.5_{-}} v'(\alpha) = \lim_{\alpha \rightarrow -0.5_{-}} \frac{4}{3} \alpha  + \frac{2}{3} = 0 = \lim_{\alpha \rightarrow -0.5_{+}} v'(\alpha), $$
			womit $v$ stetig differenzierbar ist. Nun ist $v'$ wegen 
			$$ v''(\alpha)=\frac{4}{3} > 0 \text{ für } \alpha \leq - 0.5 \text{ und } v''(\alpha) = 0 \text{ für } \alpha > - 0.5 $$
			auf der konvexen Menge $\R$ monoton und $v$ nach Satz 2.5.16 konvex.
		\end{proof}
\end{enumerate}
 
\newpage

\subsubsection{Aufgabe S. 2.3 (\textit{$13 + 4$})}

\begin{enumerate}
	\item Implementieren Sie das Schnittebenenverfahren von Kelley (vgl. Algorithmus 2.1 im Skript), das eine Approximation eines globalen Minimalpunktes eines konvexen Optimierungsproblems
		$$ P: \quad \min_{x \in \R^n} c^T x \text{ s.t. } g_i(x) \leq 0, ~ i \in I, ~Ax \leq b $$
		mit der nicht-leeren kompakten Mengen $M	^0 \coloneqq \{ x \in \R^m | Ax \leq b \}$ berechnet. Die in jeder Iteration auftretenden linearen Optimierungsprobleme sollen mit dem Solver Gurobi geläst werden.
		\begin{proof} ~\\
			\begin{figure}[h!] \centering \includegraphics[scale=0.45]{img/func} \label{fig:sub1}\end{figure} 
		\end{proof} \newpage
	\item Testen Sie Ihren Algorithmus an dem Optimierungsproblem für $\epsilon \coloneqq 10^{-7}$
		$$ P: \quad \min_{x \in \R^2} x_2 ~ \text{ s.t. } $$
		$$ x_1^2 + x_2^2 \leq 1, ~e^{x_1} - x_2 \leq 0, ~x_1 - x_2 \leq 0, ~ x_1 \in [-1, -\frac{1}{2}], ~ x_2 \in [-1, 1]. $$
		\begin{proof} ~\\
			\begin{figure}[h!] \centering \includegraphics[scale=0.51]{img/scr} \label{fig:sub2}\end{figure} 
		\end{proof}
\end{enumerate} 

\newpage

\subsubsection{Anhang}
 

\begin{theorem}
 	 Die Funktion $f \colon \R^n \rightarrow \R$ sei stetig. Dann sind die Mengen 
 	 	$$ f^\alpha_\leq = \{ x \in \R^n \colon f(x) \leq \alpha\} $$
 	 für alle $\alpha \in \R$ abgeschlossen.
 	 
	\begin{proof}
 		Sei $(x^\nu) \subseteq \R^n$ eine Folge in $f_\leq^\alpha$. Weiterhin sei
 			$$ \lim_{\nu \rightarrow \infty} f(x^\nu) \eqqcolon f(x), $$
 			für ein $x \in \R^n$. Zu zeigen ist, dass $x \in f_\leq^\alpha$, woraus die Abgeschlossenheit folgt. Es gilt:
 			$$ f(x^\nu) \leq \alpha \quad \forall \nu \in \N. $$
 		Aufgrund der Stetigkeit der Funktion $f$ folgt dann:
 			$$ f(x) = f(\lim_{\nu \rightarrow \infty} x^\nu) = \lim_{\nu \rightarrow \infty} f(x^\nu) \leq \alpha, $$
 		also $f(x) \leq \alpha$ bzw. $x \in f_\leq^\alpha$.
	\end{proof}
\end{theorem}

 
 
 
\end{document} 