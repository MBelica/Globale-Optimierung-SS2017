\documentclass[12pt]{extreport} % Schriftgröße: 8pt, 9pt, 10pt, 11pt, 12pt, 14pt, 17pt oder 20pt

%% Packages
\usepackage{scrextend}
\usepackage{amssymb}
\usepackage{amsthm}
\usepackage{booktabs}
\usepackage{caption}
\usepackage{subcaption}
\usepackage{chngcntr}
\usepackage{cmap}
\usepackage{color}
\usepackage{csquotes}
\usepackage{enumitem}
\usepackage{float}
\usepackage{graphicx}
\usepackage{hyperref}
\usepackage{ulem}
\usepackage{lmodern}
\usepackage{makeidx}
\usepackage{amsmath}
\usepackage{mathtools}
\usepackage{xpatch}
\usepackage{pgfplots}
\pgfplotsset{compat=1.12}
\usepgfplotslibrary{fillbetween}
\usepackage{amsfonts}
\usetikzlibrary{calc}	
\usetikzlibrary{matrix}	
\usepackage{fancyhdr}
\usepackage{epstopdf}



% Language Setup (English)
\usepackage[utf8]{inputenc} 
\usepackage[T1]{fontenc} 
\usepackage[english]{babel}

% Options
\makeatletter%%  
  % Linkfarbe, {0,0.35,0.35} für Türkis, {0,0,0} für Schwarz, {1,0,0} für Rot, {0,0,0.85} für Blau
  \definecolor{linkcolor}{rgb}{0,0.35,0.35}
  % Zeilenabstand für bessere Leserlichkeit
  \def\mystretch{1.2} 
  % Publisher definieren
  \newcommand\publishers[1]{\newcommand\@publishers{#1}} 
  % Enumerate im 1. Level: \alph für a), b), ...
  \renewcommand{\labelenumi}{\alph{enumi})} 
  % Enumerate im 2. Level: \roman für (i), (ii), ...
  \renewcommand{\labelenumii}{(\roman{enumii})}
  % Zeileneinrückung am Anfang des Absatzes
  \setlength{\parindent}{0pt} 
  % Für das Proof-Environment: 'Beweis:' anstatt 'Beweis.'
  \xpatchcmd{\proof}{\@addpunct{.}}{\@addpunct{:}}{}{} 
  % Nummerierung der Bilder, z.B.: Abbildung 4.1
  \@ifundefined{thechapter}{}{\def\thefigure{\thechapter.\arabic{figure}}} 
  % Chapter-Nummerierung beginnen bei (0):
  \setcounter{chapter}{0}
  % Chapter-Nummerierung
  \renewcommand\thechapter{\Roman{chapter}}
\makeatother%

% Meta Setup 
\title{Globale Optimierung - Sonderübung I}
\author{Kostorz, Belica}
\date{Sommersemester 2017}
\publishers{Karlsruher Institut für Technologie}

%% Math. Definitiones
\newcommand{\C}{\mathbb{C}}
\newcommand{\N}{\mathbb{N}}
\newcommand{\Q}{\mathbb{Q}}
\newcommand{\R}{\mathbb{R}}
\newcommand{\Z}{\mathbb{Z}}
\newcommand{\DO}[1]{\mathcal{D}\left( {#1} \right)}
\newcommand{\RO}[1]{\mathcal{R}\left( {#1} \right)}

\newtheoremstyle{named}{}{}{\normalfont}{}{\bfseries}{:}{0.25em}{#2 \thmnote{#3}}
\newtheoremstyle{nnamed}{}{}{\normalfont}{}{\bfseries}{:}{0.25em}{\thmnote{#3}}
\newtheoremstyle{itshape}{}{}{\itshape}{}{\bfseries}{:}{ }{}
\newtheoremstyle{normal}{}{}{\normalfont}{}{\bfseries}{:}{ }{}
\renewcommand*{\qed}{\hfill\ensuremath{\square}}

\theoremstyle{named}
\newtheorem{unnamedtheorem}{Theorem} \counterwithin{unnamedtheorem}{chapter}
\theoremstyle{nnamed}
\newtheorem*{unnamedtheorem*}{Theorem} 

\theoremstyle{itshape}
\newtheorem{definition}[unnamedtheorem]{Definition}

\theoremstyle{normal}
\newtheorem*{recall}{Recall}
\newtheorem*{example}{Example}
\newtheorem*{remark}{Remark}
\newtheorem*{satz}{Satz}
\newtheorem*{bemerkung}{Bemerkung}



\fancypagestyle{firststyle}
{
   \fancyhf[C]{\small Globale Optimierung - Sonderübung I - Nadine Kostorz (1972005), Martin Belica (1775706)}
   \fancyfoot[C]{}
}
%% Template
\makeatletter%
\DeclareUnicodeCharacter{00A0}{ } \pgfplotsset{compat=1.7} \hypersetup{colorlinks,breaklinks, urlcolor=linkcolor, linkcolor=linkcolor, pdftitle=\@title, pdfauthor=\@author, pdfsubject=\@title, pdfcreator=\@publishers}\DeclareOption*{\PassOptionsToClass{\CurrentOption}{report}} \ProcessOptions \def\baselinestretch{\mystretch} \setlength{\oddsidemargin}{0.125in} \setlength{\evensidemargin}{0.125in} \setlength{\topmargin}{0.5in} \setlength{\textwidth}{6.25in} \setlength{\textheight}{8in} \addtolength{\topmargin}{-\headheight} \addtolength{\topmargin}{-\headsep} \def\pulldownheader{ \addtolength{\topmargin}{\headheight} \addtolength{\topmargin}{\headsep} \addtolength{\textheight}{-\headheight} \addtolength{\textheight}{-\headsep} } \def\pullupfooter{ \addtolength{\textheight}{-\footskip} } \def\ps@headings{\let\@mkboth\markboth \def\@oddfoot{} \def\@evenfoot{} \def\@oddhead{\hbox {}\sl \rightmark \hfil \rm\thepage} \def\chaptermark##1{\markright {\uppercase{\ifnum \c@secnumdepth >\m@ne \@chapapp\ \thechapter. \ \fi ##1}}} \pulldownheader } \def\ps@myheadings{\let\@mkboth\@gobbletwo \def\@oddfoot{} \def\@evenfoot{} \def\sectionmark##1{} \def\subsectionmark##1{}  \def\@evenhead{\rm \thepage\hfil\sl\leftmark\hbox {}} \def\@oddhead{\hbox{}\sl\rightmark \hfil \rm\thepage} \pulldownheader }	\def\chapter{\cleardoublepage  \thispagestyle{plain} \global\@topnum\z@ \@afterindentfalse \secdef\@chapter\@schapter} \def\@makeschapterhead#1{ {\parindent \z@ \raggedright \normalfont \interlinepenalty\@M \Huge \bfseries  #1\par\nobreak \vskip 40\p@ }} \newcommand{\indexsection}{chapter} \patchcmd{\@makechapterhead}{\vspace*{50\p@}}{}{}{}\def\Xint#1{\mathchoice
    {\XXint\displaystyle\textstyle{#1}} {\XXint\textstyle\scriptstyle{#1}} {\XXint\scriptstyle\scriptscriptstyle{#1}} {\XXint\scriptscriptstyle\scriptscriptstyle{#1}} \!\int} \def\XXint#1#2#3{{\setbox0=\hbox{$#1{#2#3}{\int}$} \vcenter{\hbox{$#2#3$}}\kern-.5\wd0}} \def\dashint{\Xint-} \def\Yint#1{\mathchoice {\YYint\displaystyle\textstyle{#1}} {\YYYint\textstyle\scriptscriptstyle{#1}} {}{} \!\int} \def\YYint#1#2#3{{\setbox0=\hbox{$#1{#2#3}{\int}$} \lower1ex\hbox{$#2#3$}\kern-.46\wd0}} \def\YYYint#1#2#3{{\setbox0=\hbox{$#1{#2#3}{\int}$}  \lower0.35ex\hbox{$#2#3$}\kern-.48\wd0}} \def\lowdashint{\Yint-} \def\Zint#1{\mathchoice {\ZZint\displaystyle\textstyle{#1}}{\ZZZint\textstyle\scriptscriptstyle{#1}} {}{} \!\int} \def\ZZint#1#2#3{{\setbox0=\hbox{$#1{#2#3}{\int}$}\raise1.15ex\hbox{$#2#3$}\kern-.57\wd0}} \def\ZZZint#1#2#3{{\setbox0=\hbox{$#1{#2#3}{\int}$} \raise0.85ex\hbox{$#2#3$}\kern-.53\wd0}} \def\highdashint{\Zint-} \DeclareRobustCommand*{\onlyattoc}[1]{} \newcommand*{\activateonlyattoc}{ \DeclareRobustCommand*{\onlyattoc}[1]{##1} } \AtBeginDocument{\addtocontents{toc} {\protect\activateonlyattoc}} \newcommand{\RightArrow}{\xRightarrow[]{ ~ ~ }} \newcommand{\LeftArrow}{\xLeftarrow[]{ ~ ~ }} \newcommand{\rightArrow}{\xrightarrow[]{ ~ ~ }} \newcommand{\leftArrow}{\xleftarrow[]{ ~ ~ }}
	% Titlepage
	\def\maketitle{ \begin{titlepage} 
			~\vspace{3cm} 
		\begin{center} {\Huge \@title} \end{center} 
	 		\vspace*{1cm} 
	 	\begin{center} {\large \@author} \end{center} 
	 	\vspace*{-0.5cm}
	 	\begin{center} \@date \end{center} 
	 		\vspace*{7cm} 
	 	\begin{center} \@publishers \end{center} 
	 		\vfill 
	\end{titlepage} }
\makeatother%

% Create Index
\makeindex 

\begin{document}

\thispagestyle{empty}


\subsubsection{Aufgabe S.2} ~\\
Gegeben sei das Optimierungsproblem
$$ P: \quad \min f(x), \text{ s.t. } x \in M $$
mit
\begin{enumerate}
	\item $f(x) = - x^5$, $M =(- \infty, 1)$.
	\item $f(x) = 9 x_1^2 - 6 x_1 x_2^2 + x_2^4$, $M = \R^2$
	\item $f(x) = \frac{x^T A x}{\| x - b\|_2 + 1}$, mit $A \in \R^{n \times n}$ positiv definit, $b \in \R^n$ und $M = \R^n$.
\end{enumerate}
	Begründen Sie jeweils: ist $f$ koerziv auf $M$? Ist $P$ lösbar? ~\\
	\textbf{Hinweis}: Nutzen Sie für Aufgabenteil c) die Äquivalenz der Normen im $\R^n$ ~\bigskip
	
	\textit{Nach Vorlesung (Definition 1.2.37) gilt:} ~\\
	\textit{Gegeben seien eine (nicht notwendigerweise abgeschlossene) Menge $M \subseteq \R^n$ und eine Funktion $f \colon M \rightarrow \R$. Falls für alle Folgen $(x^\nu) \subseteq M$ mit $\lim_{\nu \rightarrow \infty} \| x^\nu \| \rightarrow \infty$ und alle konvergenten Folgen $(x^\nu) \subseteq M$ mit $\lim_{\nu} x^\nu \notin M$ die Bedingung}
	$$ \lim_{\nu \rightarrow \infty} f(x^\nu) = + \infty $$
	\textit{gilt, dann heißt $f$ koerziv auf $M$.} ~\\
	
	\begin{enumerate}
		\item $f(x) = - x^5$, $M =(- \infty, 1)$: ~\medskip
			\begin{proof}
				Behauptung: $f$ ist nicht koerziv auf $M$. ~\\
				Beweis: Sei $x^\nu = 1 - \frac{1}{\nu} \subseteq M, \nu \in \N$. Es gillt
				$$ \lim_{\nu \rightarrow \infty} x^\nu = \lim_{\nu \rightarrow \infty} 1 - \frac{1}{\nu} = 1 \notin M $$
				Da $f(x^\nu) = - \left( 1 - \frac{1}{\nu} \right)^5 \rightarrow -1 < \infty \Rightarrow[Def. 1.2.37]{} f$ ist nicht koerziv. ~\\
				
				Behauptung: $P$ ist nicht lösbar auf $M$. ~\\
				Beweis: $f'(x) = - 5 x^4 \leq 0 \Rightarrow f$ ist monoton fallend. Da $f'(x) < 0$ für $x > 0$:
				$$ f(x) \geq -1^5, $$
				aber $1 \notin M \Rightarrow$ Behauptung.
			\end{proof}
		\item $f(x) = 9 x_1^2 - 6x_1 x_2^2 + x_2^4$, $M = \R^2$
			\begin{proof}
				Behauptung: $f$ ist nicht koerziv.
				Beweis: Betrachte $x^\nu = \left( \frac{1}{3} \nu^2, v \right)$. Damit gilt
				$$ \lim_{\nu \rightarrow \infty} \left( |x^\nu| \right) = \left| \frac{1}{3} r^2 \right| + \left| r \right| \infty \text{ und } \lim_{\nu \rightarrow \infty} f(x^\nu) = \lim \left( 3 \left( \frac{1}{3} \nu^2 - \nu^2 \right) \right)^2 = 0 $$
				Betrachtet man die Lösbarkeit des Problems, so gilt dass $f(x) \geq 0$ und für $\overline{x} = (0, 0)$:
				$$ f(\overline{x}) = 0 $$
			\end{proof}
		\item Behauptung $f$ ist koerziv auf $M = \R^n$. ~\\
			Sei $(x^\nu) \subseteq M$ mit $\| x^\nu \| \rightarrow \infty$. Aus Übung 2 ist bekannt, dass 
			$$ x^T A x = \| x \|_{\tilde{A}}^2 $$
			mit $\tilde{A}$ aus Übung 2. Wegen der Äquivalenz der Normen im $\R^n$ existiert ein $c > 0$ sodass:
			$$ \| x \|_2 \leq c \| x \|_{\tilde{A}} $$
			$\Rightarrow f(x) = \frac{x^T A x }{\| x - b \|_2 + 1} = \frac{\| x \|_{\tilde{A}}^2}{\| x - b \|_2 + 1}$
			$$ \Rightarrow \frac{\| x \|_{\tilde{A}}^2}{\| x - b \|_2 + 1} \geq \frac{\| x \|_{\tilde{A}}^2}{c \| x\|_{\tilde{A}} + \| b \|_2 + 1} $$
			daher gilt insbesondere 
			$$ f(x^\nu) = \frac{\| x^\nu \|_{\tilde{A}^2}}{c \| x^\nu \|_{\tilde{A}} + \| b \|_2 + 1} \rightarrow \infty $$
			für $\nu \rightarrow \infty$. Da $M = \R^n$ abgeschlossen ist, ist $f$ nach Definition 1.2.23 koerziv. Da außerdem $M$ nicht leer und abgeschlossen und $f$ koerziv und stetig:
			$$ \Rightarrow P \text{ lösbar nach 1.2.30} $$
	\end{enumerate}
	
\subsubsection{Aufgabe S.2} ~\\

\\begin{enumerate}
	\item Es ist
			$$ P_{epi}: \quad \min_{x, \alpha) \in \R^2 \times \R} F(\alpha) \text{ s.t. } f(x) \leq \alpha $$
		Definiere $F(\alpha) \coloneqq e^{\alpha}$, $f(x) = - \min \big\{ - x-1 - 3, - \left| x_2 - 4 \right|, x_1+ x_2 -20 \}$. Damit ist
			$$ P_{epi}: \quad \min_{(x, \alpha) \in \R^2 \times \R} e^\alpha { s.t. } - \min \{ - x_1 - 3, - \left| x_2 - 4 \right|, x_1 + x_2 -20 \} \leq \alpha $$
	\item Übung 3.2.: $\min e^\alpha$ ist äquivalent zu $\min \alpha$ ~\\
		$$ - \min \{ \dotsc \} \leq \alpha \iff \min \{ \dotsc \} \geq - \alpha$$
		Aufspalten des Minimums liefert:
		$$ P_{lin} \min_{(x, \alpha) \in \R^2 \times \R} \alpha \text{ s.t. } $$
		$$ -x_1 - 3 \geq - \alpha, - x_2 + 4 \geq - \alpha, x_2 -4 \geq - \alpha, x_1 + x_2 - 20 \geq -\alpha $$
	
\end{enumerate}

	
\end{document}